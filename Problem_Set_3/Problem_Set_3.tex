\documentclass[12pt,onecolumn]{article}

%%%%%%%%%%%%%%%%%%%%%%%%%%%%%%%%%%%
%          				PACKAGES  				              %
%%%%%%%%%%%%%%%%%%%%%%%%%%%%%%%%%%%
\usepackage[margin=1.5in]{geometry}
\usepackage{authblk}
%\usepackage[latin1]{inputenc}
\usepackage[utf8]{inputenc}
\usepackage{placeins}
\usepackage{amsfonts}
\usepackage{comment}
\usepackage{a4wide,graphicx,color}
\usepackage[colorlinks=true,linkcolor=black,urlcolor=blue,citecolor=blue]{hyperref}
\usepackage{amsmath}
\usepackage{bbm}
\usepackage[table]{xcolor}
\usepackage{setspace}
\usepackage{booktabs}
\usepackage{dcolumn}
\usepackage{color,soul}
\usepackage{threeparttable}
\usepackage[capposition=top]{floatrow}
\usepackage[labelsep=period]{caption}
\usepackage{caption}
\usepackage{subcaption}
\usepackage{lscape}
\usepackage{pdflscape}
\usepackage{multicol}
\usepackage[bottom]{footmisc}
\setlength\footnotemargin{5pt}
\usepackage{longtable}
\usepackage{chronosys}
\catcode`\@=11
\def\chron@selectmonth#1{\ifcase#1\or Jan\or Feb\or Mar\or Apr\or May\or Jun\or Jul\or Aug\or Sep\or Oct\or Nov\or Dec\fi}

%% BibTeX settings
\usepackage{natbib}
\bibliographystyle{apalike}
\bibpunct{(}{)}{,}{a}{,}{,}

%% markup commands for code/software
\let\code=\texttt
\let\pkg=\textbf
\let\proglang=\textsf
\newcommand{\file}[1]{`\code{#1}'}
\newcommand{\email}[1]{\href{mailto:#1}{\normalfont\texttt{#1}}}
\urlstyle{same}

%% paragraph formatting
\renewcommand{\baselinestretch}{1}

%% \usepackage{Sweave} is essentially
\RequirePackage[T1]{fontenc}
\RequirePackage{ae,fancyvrb}
\DefineVerbatimEnvironment{Sinput}{Verbatim}{fontshape=sl}
\DefineVerbatimEnvironment{Soutput}{Verbatim}{}
\DefineVerbatimEnvironment{Scode}{Verbatim}{fontshape=sl}
\newenvironment{Schunk}{}{}

% Defines columns for tables
\usepackage{array}
\newcolumntype{L}[1]{>{\raggedright\let\newline\\\arraybackslash\hspace{0pt}}m{#1}}
\newcolumntype{C}[1]{>{\centering\let\newline\\\arraybackslash\hspace{0pt}}m{#1}}
\newcolumntype{R}[1]{>{\raggedleft\let\newline\\\arraybackslash\hspace{0pt}}m{#1}}


\usepackage{bbm}
\usepackage{enumitem}
%%%%%%%%%%%%%%%%%%%%%%%%%%%%%%%%%%%
%     			TITLE, AUTHORS AND DATE    			  %
%%%%%%%%%%%%%%%%%%%%%%%%%%%%%%%%%%%

\title{Problem Set 3}
\usepackage{etoolbox}
\makeatletter
\providecommand{\subtitle}[1]{% add subtitle to \maketitle
  \apptocmd{\@title}{\par {\large #1 \par}}{}{}
}
\makeatother
\subtitle{Econ 4676: Big Data and Machine Learning for Applied Economics}
\author{{\bf Due Date}: October 1 at 1:00 pm}
\date{}
\date{The repo link to create your submission is \url{https://classroom.github.com/g/gzR7whZ3}}

\begin{document}
\maketitle

\section{Theory Exercises: MLE and Spatial Econometrics}

\begin{enumerate}
  
  \item {\it Binary Response Online Updating}. The problem is simple but yet complicated. Online updating is essential because it breaks the storage barrier and helps with computation. Assume that a new observation arrives, and instead of refitting the entire model, you want to update your binary model estimates. Show that the contribution made by the observation $i$ to the likelihood function is

\begin{align}
l(y,\beta) = \sum_{i=1}^N \left( y_i log F(x'_i\beta) + (1-y_i) log (1-F(x'_i\beta)  \right))
\end{align}
  is globally concave with respect to $\beta$ if the function $F$ is such that $F(-x)=1-F(x)$, and if its first derivative $f$, and its second derivative $f'$ satisfy the condition:
  \begin{align}
  f'(x)F(x)-f^2(x)<0
  \end{align}
  for all real finite $x$. Show that this condition is satisfied by the logistic function $\Lambda(x)=\frac{1}{1+e^{-x}}=\frac{e^{x}}{1+e^{x}}$

\item In many sub-fields of economics, like finance, it is common that the tails of the noise distribution are much heavier than the standard Gaussian tails. One way to model this is to use a $t-distribution$. Consider the following model: $y_i = \alpha +\beta x_i +\epsilon, i=1,...,N,$, i.e., a model with a constant and a single regressor. But now $\epsilon / \sigma \sim_{iid} t_v$, and $E(X\epsilon)=0$ and $v$ are the degrees of freedom.
  \begin{enumerate}
    \item Write down the log-likelihood, using the explicit formula for the density of the $t-distribution$.
    \item Find the first derivatives of this log-likelihood with respect to the parameters $\alpha$, $\beta$, $\sigma^2$, and $v$.
    \item Assume that $\sigma^2$ and $v$ are known. Can you solve for the maximum likelihood estimator of $\alpha$ and $\beta$? Do they match the least-square estimators? If not, why and how do they differ?
    \item Using the software of your choice, write a function that takes as arguments some data (y,x) and outputs the MLE estimates from a $t-distribution$.
    \item Simulate some data that follow the model described above, and answer the following questions:
    \begin{enumerate}
      \item How well does the MLE recover the parameters?
      \item Does it get better as N grows?
      \item What about when the variance of X increases?
      \item How does it compare relative to a naive OLS estimator?
    \end{enumerate}


\item Each student in the class has an account in AWS. You can access it with your \texttt{@uniandes.edu.co} user. Set up an AWS instance and install a software of your choice. Attach screenshots of the virtual machine running. Some suggestions to install:
  \begin{enumerate}
    \item \texttt{R} with \texttt{RStudio}
    \item \texttt{JupyterLab}
    \item \texttt{Python}
  \end{enumerate}
  
  \item Repeat the simulation in \texttt{e} using a parallel or distributed approach in AWS
  \begin{enumerate}
    \item Did you get the same results as above? If not, why?
    \item How did you handle the \texttt{seed} in this context?
    \item Was there a computational time gain? Report the differences.
    \item Indicate who in your team ran the simulations. (I'll check AWS usage)
  \end{enumerate}

    \end{enumerate}
  

  \item Suppose you have the following spatial model $y=\rho W y + X\beta + WX\theta  +\epsilon$ with $|\rho|<1$  this is sometimes known as the Spatial Durbin Model.
  \begin{enumerate}
    \item First, consider the following scenario  $\beta=\theta=0$. 
    \begin{enumerate}
      \item Write the Likelihood function. Can you find a closed form for the parameter estimators? Don't forget to be specific on the assumptions you make.
      \item Suppose instead you use MCO, would you obtain the same estimates? 
    \end{enumerate}  
    \item Now consider that $\rho=0$, and let's proceed as before:
    \begin{enumerate}
      \item Write the Likelihood function. Can you find a closed form for the parameter estimators? Don't forget to be specific on the assumptions you make.
      \item Suppose instead you use MCO, would you obtain the same estimates? 
  \end{enumerate}  
  \end{enumerate}  
  \end{enumerate}  

\section{Empirical Problems}


The main objective of this section is to apply the concepts we learned using ``real" world data. With these, I also expect that you sharpen your data collection and wrangling skills. Finally, you should pay attention to your writing.

I encourage you to turn the following section of the problem set in a way that resembles a paper. As such, I expect graphs, tables, and writing to be as neat as possible. You can write it in Spanish or English, and either language is acceptable. For students in the Ph.D., it would be a good practice to do it in English.
These parts also involve a lot of coding. Don't forget to upload everything to your repository and follow the template repository. 



\subsection{Getting to know Evanston, IL}

This part of the problem set involves a series of spatial data sets on the City of Evanston, IL. All the relevant data sets are in the \texttt{data} folder.  The \texttt{evanston\_parcel\_data.csv} file contains parcel level data from Evanston. The data was retrieved from the county assessors' office.\footnote{For more info and variable definitions, check \url{https://tinyurl.com/y6y6bhat} and \url{https://datacatalog.cookcountyil.gov/stories/s/p2kt-hk36}. Note how they use \texttt{Gitlab} for version control and ML for prediction}. The first objective is to showcase your mapping abilities. The second objective is to use the tools studied in class to model and predict assessment values using all the provided information.

\begin{enumerate}
  \item {\it ``Mapping the field''}
  \begin{enumerate}
    \item Begin by creating a map that includes census area identifiers (census blocks, census tracts),  major infrastructure layers (train line, roads, etc.), and Lake Michigan shoreline.
    \item Match the parcel data to the block level file and calculate (i) average assessment values and (ii) building area to floor area at the  block level.
    \item Describe your results. In your description, include a side by side map using the map you created in part (a) that includes the information you generated in (b)
  \end{enumerate}
  
  
  \item In Problem Set 1, we focused on obtaining the best in-sample fit. Here we shift our focus to prediction.
  \begin{enumerate}
     \item The data set include multiple variables that can help predict the assessment value of a property.  Describe those that you will use in your predictive exercise. I leave it to you to decide which variables to include and their functional form. Don't forget to include  the ``spatial variables,'' i.e. distances to major infrastructures, that you created in (1).
  \item We will continue exploring linear models of the form
    
    \begin{equation}
      y = X\beta +u
    \end{equation}

  Where $Y$ is the assessment price, and X  is a matrix with the variables you chose to predict $Y$. Our approach will be very agnostic, the objective is to minimize the prediction error.
  \begin{enumerate}
    \item Start with a model that only includes a constant. Then estimate more complex models. These models can include more variables, interactions, transformations, etc. I expect that you estimate at least 10 models.
    \begin{enumerate}
      \item Describe and explain how and why you built these models.
      \item Report, describe, and compare the prediction error. 
      \item Explain how you calculated the prediction error.
    \end{enumerate}
    \item Discuss the model with the lowest prediction error.
  \end{enumerate}
  
  \item Estimate the  model with the lowest prediction error in (b) adding spatial structure to it. You can use any of the spatial models that we talked in class or any other that you deem appropriate. For example, you can run $y=WX\beta +u$. 
  \begin{enumerate}
    \item Discuss how you defined proximity between observations. 
    \item Did the spatial structure improve your model? 
    \item Can you specify a spatial model that improves upon the ``best'' in part (b)?
  \end{enumerate}
  \item {\it LOOCV}. With your preferred predicted model (the one with the lowest average prediction error) perform the following exercise:
    \begin{enumerate}
      \item Write a loop for  the first 1,000 observations ($i = 1$ to $i = 1,000$). The loop should do the following:
      \begin{itemize}
        \item Estimate the regression model using all but the $i-th$ observation.
        \item Calculate the prediction error for the $i-th$ observation, i.e. ($y_i-\hat y_i$)
        \item Calculate the average of the numbers obtained in the previous step to obtain the average mean square error. This is known as the Leave-One-Out Cross-Validation (LOOCV) statistic.
      \end{itemize}
        \item Compute the leverage statistic for each observation. Show analytically and empirically that the leverage statistic can be used for the computation of the LOOCV statistic.
        \item Use the statistic derived in the previous point to calculate the LOOCV statistic.
\end{enumerate}
\end{enumerate}
  \end{enumerate}
  {\it {\bf Note}: If you find that the data set is too large for spatial models you have a couple of options. As a first option, use a random subsample and operate on the smaller sample. As a second (and recommend) option, use  AWS or Azure and take advantage of the cloud.}





\end{document}
