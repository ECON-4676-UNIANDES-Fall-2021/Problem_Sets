\documentclass[12pt,onecolumn]{article}

%%%%%%%%%%%%%%%%%%%%%%%%%%%%%%%%%%%
%          				PACKAGES  				              %
%%%%%%%%%%%%%%%%%%%%%%%%%%%%%%%%%%%
\usepackage[margin=1.5in]{geometry}
\usepackage{authblk}
%\usepackage[latin1]{inputenc}
\usepackage[utf8]{inputenc}
\usepackage{placeins}
\usepackage{amsfonts}
\usepackage{comment}
\usepackage{a4wide,graphicx,color}
\usepackage[colorlinks=true,linkcolor=black,urlcolor=blue,citecolor=blue]{hyperref}
\usepackage{amsmath}
\usepackage{bbm}
\usepackage[table]{xcolor}
\usepackage{setspace}
\usepackage{booktabs}
\usepackage{dcolumn}
\usepackage{color,soul}
\usepackage{threeparttable}
\usepackage[capposition=top]{floatrow}
\usepackage[labelsep=period]{caption}
\usepackage{caption}
\usepackage{subcaption}
\usepackage{lscape}
\usepackage{pdflscape}
\usepackage{multicol}
\usepackage[bottom]{footmisc}
\setlength\footnotemargin{5pt}
\usepackage{longtable}
\usepackage{chronosys}
\catcode`\@=11
\def\chron@selectmonth#1{\ifcase#1\or Jan\or Feb\or Mar\or Apr\or May\or Jun\or Jul\or Aug\or Sep\or Oct\or Nov\or Dec\fi}

%% BibTeX settings
\usepackage{natbib}
\bibliographystyle{apalike}
\bibpunct{(}{)}{,}{a}{,}{,}

%% markup commands for code/software
\let\code=\texttt
\let\pkg=\textbf
\let\proglang=\textsf
\newcommand{\file}[1]{`\code{#1}'}
\newcommand{\email}[1]{\href{mailto:#1}{\normalfont\texttt{#1}}}
\urlstyle{same}

%% paragraph formatting
\renewcommand{\baselinestretch}{1}

%% \usepackage{Sweave} is essentially
\RequirePackage[T1]{fontenc}
\RequirePackage{ae,fancyvrb}
\DefineVerbatimEnvironment{Sinput}{Verbatim}{fontshape=sl}
\DefineVerbatimEnvironment{Soutput}{Verbatim}{}
\DefineVerbatimEnvironment{Scode}{Verbatim}{fontshape=sl}
\newenvironment{Schunk}{}{}

% Defines columns for tables
\usepackage{array}
\newcolumntype{L}[1]{>{\raggedright\let\newline\\\arraybackslash\hspace{0pt}}m{#1}}
\newcolumntype{C}[1]{>{\centering\let\newline\\\arraybackslash\hspace{0pt}}m{#1}}
\newcolumntype{R}[1]{>{\raggedleft\let\newline\\\arraybackslash\hspace{0pt}}m{#1}}


\usepackage{bbm}
\usepackage{enumitem}
%%%%%%%%%%%%%%%%%%%%%%%%%%%%%%%%%%%
%     			TITLE, AUTHORS AND DATE    			  %
%%%%%%%%%%%%%%%%%%%%%%%%%%%%%%%%%%%

\title{Final Project Guidelines}
\usepackage{etoolbox}
\makeatletter
\providecommand{\subtitle}[1]{% add subtitle to \maketitle
  \apptocmd{\@title}{\par {\large #1 \par}}{}{}
}
\makeatother
\subtitle{Econ 4676: Big Data and Machine Learning for Applied Economics}
\author{{\bf Due Date}: December 10 at 5:00 pm}
\date{}
%\href{https://github.com/ECON-4676-UNIANDES}{ECON-4676}}

\begin{document}
\maketitle

%\section{}

The purpose of the Final Project is for you to culminate the learning achieved in the course by describing your understanding and application of knowledge 


Focus of the Final Paper

Select a topic of interest in organizational behavior that you would like to explore with additional research (must use at least five references).



There are two possibilities for the research paper: (a) replication of an approved urban paper, or (b) an empirical paper with original research.

define a topic and outline an action plan. It will have three stages: 

\begin{enumerate}
\item First submission. This submission should be a brief statement of what you plan to do.  I expect that you provide sufficient background so that the reader can understand the problem/research question you want to address. It should include an outline of how you plan to accomplish it, including the data you have/or plan to acquire, the methods, and any other relevant information you deem necessary.
It should be at most two pages in length. I'll provide feedback and will ask for changes if needed. This is worth 5\% of your final grade.
\end{enumerate}

assignment is to write a 10-15 page project proposal. The aim of a proposal is to define a topic and outline a research plan. While many proposals promise that multiple papers will result, you will focus on a project that will produce a single paper. When you have completed your proposal, you will possess a blueprint for a research paper. Then "all" that remains to be done is the execution...


\end{document}
