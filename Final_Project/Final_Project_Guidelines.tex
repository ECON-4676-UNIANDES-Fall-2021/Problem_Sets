\documentclass[12pt,onecolumn]{article}

%%%%%%%%%%%%%%%%%%%%%%%%%%%%%%%%%%%
%          				PACKAGES  				              %
%%%%%%%%%%%%%%%%%%%%%%%%%%%%%%%%%%%
\usepackage[margin=1.5in]{geometry}
\usepackage{authblk}
%\usepackage[latin1]{inputenc}
\usepackage[utf8]{inputenc}
\usepackage{placeins}
\usepackage{amsfonts}
\usepackage{comment}
\usepackage{a4wide,graphicx,color}
\usepackage[colorlinks=true,linkcolor=black,urlcolor=blue,citecolor=blue]{hyperref}
\usepackage{amsmath}
\usepackage{bbm}
\usepackage[table]{xcolor}
\usepackage{setspace}
\usepackage{booktabs}
\usepackage{dcolumn}
\usepackage{color,soul}
\usepackage{threeparttable}
\usepackage[capposition=top]{floatrow}
\usepackage[labelsep=period]{caption}
\usepackage{caption}
\usepackage{subcaption}
\usepackage{lscape}
\usepackage{pdflscape}
\usepackage{multicol}
\usepackage[bottom]{footmisc}
\setlength\footnotemargin{5pt}
\usepackage{longtable}
\usepackage{chronosys}
\catcode`\@=11
\def\chron@selectmonth#1{\ifcase#1\or Jan\or Feb\or Mar\or Apr\or May\or Jun\or Jul\or Aug\or Sep\or Oct\or Nov\or Dec\fi}

%% BibTeX settings
\usepackage{natbib}
\bibliographystyle{apalike}
\bibpunct{(}{)}{,}{a}{,}{,}

%% markup commands for code/software
\let\code=\texttt
\let\pkg=\textbf
\let\proglang=\textsf
\newcommand{\file}[1]{`\code{#1}'}
\newcommand{\email}[1]{\href{mailto:#1}{\normalfont\texttt{#1}}}
\urlstyle{same}

%% paragraph formatting
\renewcommand{\baselinestretch}{1}

%% \usepackage{Sweave} is essentially
\RequirePackage[T1]{fontenc}
\RequirePackage{ae,fancyvrb}
\DefineVerbatimEnvironment{Sinput}{Verbatim}{fontshape=sl}
\DefineVerbatimEnvironment{Soutput}{Verbatim}{}
\DefineVerbatimEnvironment{Scode}{Verbatim}{fontshape=sl}
\newenvironment{Schunk}{}{}

% Defines columns for tables
\usepackage{array}
\newcolumntype{L}[1]{>{\raggedright\let\newline\\\arraybackslash\hspace{0pt}}m{#1}}
\newcolumntype{C}[1]{>{\centering\let\newline\\\arraybackslash\hspace{0pt}}m{#1}}
\newcolumntype{R}[1]{>{\raggedleft\let\newline\\\arraybackslash\hspace{0pt}}m{#1}}


\usepackage{bbm}
\usepackage{enumitem}
%%%%%%%%%%%%%%%%%%%%%%%%%%%%%%%%%%%
%     			TITLE, AUTHORS AND DATE    			  %
%%%%%%%%%%%%%%%%%%%%%%%%%%%%%%%%%%%

\title{Final Project Guidelines}
\usepackage{etoolbox}
\makeatletter
\providecommand{\subtitle}[1]{% add subtitle to \maketitle
  \apptocmd{\@title}{\par {\large #1 \par}}{}{}
}
\makeatother
\subtitle{Econ 4676: Big Data and Machine Learning for Applied Economics}
\author{{\bf Due Date}: December 10 at 6:00 pm}
\date{}
%\href{https://github.com/ECON-4676-UNIANDES}{ECON-4676}}

\begin{document}
\maketitle

%\section{}

The purpose of the final project is for you to showcase the skills that you acquired in this course. You can choose three routes for the project (a) a novel research paper, (b) replicate a paper that you find interesting, or (c) deploy a machine learning model in a web application that allows practical business decisions based on data. If you choose either (b) or (c) you should  talk to me first.

There are three stages for the final project:

\begin{enumerate}
\item First submission due on October 22nd at 6 pm. This submission should be a brief statement of what you plan to do.  I expect that you provide sufficient background so that the reader can understand the problem/research question you want to address. It should include an outline of how you plan to accomplish it, including the data you have/or plan to acquire, the methods, and any other relevant information you deem necessary.
It should be at most 2 (two) pages in length. I'll provide feedback and will ask for changes if needed. This is worth 5\% of your final grade.

\item Presentation of your work. Presentations will be in class, November 30th, December 2nd, and 3rd. The order would be randomly assigned. It should not be longer than 15 minutes. You have to treat this as a seminar/stakeholder presentation. For presentation tips, you can check this \href{https://ignaciomsarmiento.github.io/teaching/seminar/Tips_Presentation_PEG.pdf}{document} or  \href{https://ignaciomsarmiento.github.io/teaching/Tesis.html}{last semester's seminar class}. This is worth 10\% of your grade.


\item Final  submission  due on December 10th at 6 pm. The final document should not be longer than 5 (five) pages (not including the title page with abstract, and references). When writing up the document, it should contain the following:
\begin{itemize}
  \item Title
  \item Abstract (200 words limit)
  \item Introduction. It should contain at least: the problem/research question clearly defined,  antecedents of your work, your value-added (i.e. why your project is interesting/novel/different), and a preview of  results and takeaways.
  \item Data. Treat this section as an opportunity to present a compelling narrative to justify or defend your data choices, walk the reader through your reasoning of how you think you got the right data for the task, describe it accordingly with descriptive stats, graphs, etc.
  \item Model. Present the model you are using.  Be sure to argue why this is the best model for your task. Did you apply other models, and this is the best at predicting? Is this the only model that you can use? Etc. 
  \item Results. Here you should present your results. I understand that a semester is a short time to have a full paper, so preliminary results are fine. 
  \item Conclusions and recommendations. In this section, you should state the main takeaways of your work.
\end{itemize}
This is worth 15\% of your grade.
\end{enumerate}

\end{document}
